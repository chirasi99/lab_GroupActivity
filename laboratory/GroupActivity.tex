\documentclass[a4paper,12pt]{article}
\begin{document}
\title{Computer Technology}
\author{C.A.M.Walopla, W.M.T. Mahehsika}
\date{July 2021}
\section{First Generation Computer} 
Generations timeline - 1940s 1950s \\ 
Evolving hardware - Vacuum tubes based \\ 
Vacuum tube - an electroic gadget that controls the progression of electrons in a vacuum utilized as a switch, intensifier, or show screen in numerous more established model radios, Televisions, PCs, and so on.

\begin{enumerate}
\item Main electronic component - Vacuum tube
\item Main memory - magnetic drums and magnetic tapes
\item Programming language - machine language
\item Power - consume a lot of electricity and generate a lot of heat.
\item Speed and size- very slow and very large in size (often taking up entire room)
\item Input/output devices- punched cards and paper tape.
\item Esxsiamples- ENIAC,UNIVAC1, IBM 650, IBM 701, etc..
\item Quantity - there were about 100 different vacuum tube computers produced between 1942 and 1963.
\end{enumerate}

\textbf{Advantages} \\ 
\begin{itemize}
\item It utilized vacuum tubes which are the solitary electronic part accessible during the those days.
\item These PCs could compute in milliseconds.
\end{itemize}

\textbf{Disadvantages}
\begin{itemize}
\item These were extremely large in size, weight was around 30 tones.
\item These PCs depended on vacuum tubes
\item These PCs were expensive.
\item It could store just a limited quantity of data because of the presence of attractive drums.
\item As the innovation of original PCs includes vacuum tubes, so another disservice of these PCs was, vaccum tubes require a huge cooling framework.
\item Less work effectiveness.
\item Restricted programming abilities and punch cards were utilized to take input.
\item Huge measure of energy utilization.
\item Not solid and consistent support is required.
\end{itemize}

\section{Second Generation Computer}
Generations timeline - 1959- 1965 \\
Evolving hardware - Transistors based \\
Transistors - A transistor computer, now often called a second generation computer, is a computer which uses discrete transistors instead of vacuum tubes. The first generation of electronic computers used vacuum tubes, which generated large amounts of heat, were bulky and unreliable.
\begin{enumerate}
\item Main electronic component - Transistors
\item Main memory - magnetic cores
\item Programming language -assembly language and high-level programming languages like FORTRAN, COBOL were used
\item power - Second-generation computers featured circuit boards filled 
\item Input/output devices - Punched cards and magnetic tape
\item Esxsiamples- IBM 1620, IBM 7094, CDC 1604, CDC 3600, UNIVAC 1108, etc..
\end{enumerate}

\textbf{Advantages} \\
\begin{itemize}
\item Smaller in size as compared to the first generation computer.
\item The second-generation computers were more reliable.
\item Wider commercial use.
\item Better portability as compared to the first generation computers.
\item Better speed and could calculate data in microseconds.
\item Used faster peripherals like tape drives, magnetic disk etc.
\item Used assembly language as well as machine language.
\item Accuracy improved.
\end{itemize}

\textbf{Disadvantages}
\begin{itemize}
\item The cooling system was required
\item Constant maintenance required.
\item Commercial production was difficult.
\item Only used for specific purposes.
\item Costly and not versatile.
\item Punch cards were used for input.
\end{itemize}



\end{document}
